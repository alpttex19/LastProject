%!TEX program = pdflatex
\documentclass[UTF8]{ctexart}
\usepackage{graphicx}
\usepackage{geometry}
\usepackage{multirow}
\usepackage{multicol}
\usepackage{amsmath}
\usepackage{amsthm}
\usepackage{amssymb}
\usepackage{bm}
\usepackage{amsfonts}
\usepackage{fancyhdr}
\usepackage{mathrsfs}
\usepackage{caption}
\usepackage{enumerate}
\usepackage{subfigure}
\usepackage{float}
\usepackage{circuitikz}
\geometry{papersize={21cm,29.7cm}}
\geometry{left=2.54cm,right=2.54cm,top=3.18cm,bottom=3.18cm}
\title{天气信息查询应用设计与实现}
\author{自12班\quad 阿拉帕提·吐尔逊\quad 2019010363}
\date{\today}
\pagestyle{fancy}
\lhead{\today}
\chead{}
\rhead{天气信息查询应用设计与实现}
\lfoot{Python程序设计}
\cfoot{\thepage}
\rfoot{智能传感与检测技术}
\renewcommand{\headrulewidth}{0.4pt}
\renewcommand{\headwidth}{\textwidth}
\renewcommand{\footrulewidth}{0pt}
\newtheorem*{theorem*}{定理}
\newtheorem{theorem}{定理}
\newtheorem*{corollary*}{推论}
\begin{document}
\maketitle

\section{实验目标}

设计并实现一个天气信息查询应用,通过调用国内和国际天气API接口,
提供用户友好的图形用户界面,支持实时天气查询、图表展示和用户收藏功能。

\section{实验步骤和内容}

\subsection{API接口选择}
选择并注册国内和国际天气API接口,用于获取实时天气信息。在本实验中,选择的国内API为高德地图天气API,
\textcolor{blue}{https://restapi.amap.com/v3/weather/weatherInfo?key={}\&city={}\&extensions=all},
国际API为OpenWeatherMap API,
\textcolor{blue}{http://api.openweathermap.org/data/2.5/forecast?lat={lat}\&lon=\\{lon}\&appid={appid}}。

\subsection{系统设计架构}

设计一个基于tkinter的图形用户界面,通过requests库调用API接口获取天气数据。核心模块包括Weather、WeatherGet、GlobalWeather、GlobalWeatherGet、MyFavoriteCity以及WeatherGUI。
\begin{itemize}
   \item Weather类表示天气信息的数据结构,包括城市、日期、温度等信息。
   \item WeatherGet和GlobalWeatherGet类负责从API获取天气信息,并将其转化为Weather和GlobalWeather对象。
   \item MyFavoriteCity类用于管理用户收藏的城市和相应的天气信息。
   \item WeatherGUI类是图形用户界面的核心,负责展示天气信息、图表绘制和用户交互。
\end{itemize}

\subsection{核心模块详解}

\begin{enumerate}
   \item `Weather`类
   \item `WeatherGet`类
   \item `GlobalWeather`类
   \item `GlobalWeatherGet`类
   \item `MyFavoriteCity`类
   \item `WeatherGUI`类
\end{enumerate}

\subsection{交互流程}
\begin{enumerate}
   \item 国内天气查询:
         \begin{itemize}
            \item 用户打开应用,选择国内天气查询。
            \item 在城市下拉框中选择城市。
            \item 选择日期。
            \item 单击“查询”按钮,应用调用高德地图API获取天气信息。
            \item 显示查询结果,包括实时温度、湿度、风向等。
            \item 用户可以点击“添加收藏”按钮将城市加入收藏列表。
         \end{itemize}
   \item 国际天气查询:
         \begin{itemize}
            \item 用户选择国际天气查询。
            \item 在国家下拉框中选择国家。
            \item 选择城市。
            \item 选择时间。
            \item 单击“查询”按钮,应用调用OpenWeatherMap API获取天气信息。
            \item 显示查询结果,包括温度、湿度、风向等。
            \item 用户可以点击“添加收藏”按钮将城市加入收藏列表。
         \end{itemize}
   \item 查看收藏列表:
         \begin{itemize}
            \item 用户点击“收藏列表”按钮。
            \item 显示用户已收藏的城市列表。
            \item 用户可以选择查看某个城市的详细天气信息。
         \end{itemize}
   \item 语言切换:
         \begin{itemize}
            \item 用户可以通过切换语言按钮选择中文或英文。
            \item 应用根据选择切换显示语言。
         \end{itemize}
\end{enumerate}


\subsection{实验扩展}

\begin{itemize}
   \item 添加用户登录功能,实现用户天气信息的个性化管理。
   \item 实现天气预报功能,提供未来几天的天气趋势。
   \item 使用数据库存储用户收藏信息,实现数据持久化。
\end{itemize}
以上实验大纲旨在指导学生完成一个简单的天气查询应用,同时提供了一些拓展性的思考。学生可以根据实际情况和实验要求进行适度调整和扩展。

\end{document}